%! Author = joaopintosp
%! Date = 8/24/24

% Preamble
\documentclass[11pt, a4paper]{article}
\usepackage[portuguese]{babel}
\usepackage{authblk}

% Packages
\usepackage{amsmath}
\usepackage{graphicx, subcaption}
\usepackage[margin=1in]{geometry}
\usepackage{indentfirst}
\usepackage{setspace}
\usepackage{emptypage}
\onehalfspacing
\setlength{\parindent}{1.25cm}
\setlength{\parskip}{6pt}
\usepackage{booktabs}
\usepackage{lipsum}
\usepackage{wrapfig2}

\author[1]{João Sá Pereira}
\author[2]{Vasco Reis}
\affil[1]{FEUP}
\affil[2]{IST}
\title{Pycharm e \LaTeX}
\date{\today}


% Document
\begin{document}

    \maketitle
    \tableofcontents
    \listoffigures
    \listoftables
    \section{Introdução}\label{sec:introducao}

    \lipsum[1]

    \subsection{Motivo}\label{subsec:motivo}

    Isto é apenas uma introdução para ver se gosto ou não de utilizar o Pycharm com\LaTeX.
    Pode ser que se torne a minha nova ferramenta de trabalho, sendo que em 2025 começarei a minha dissertação de mestrado.
    \lipsum[1]

    \subsection{Ortografia}\label{subsec:ortografia}

    Será que o Pycharm corrige a ortografia?
    Pelos vistos, sim!
    O Pycharm corrige a ortografia.
    \lipsum[2]

    \paragraph{Parágrafo da Introdução} \lipsum[3]
    
    \subsection{Figuras e Tabelas}\label{subsec:figuras-e-tabelas}

    Aqui está uma tabela:

    \begin{table}[htb!]
        \centering
        \begin{tabular}{lrr}
\toprule
name & age & height \\
\midrule
FREDERICO & 24 & 170.8 \\
DONATELLO & 45 & 177.7 \\
\bottomrule
\end{tabular}

        \caption{Uma tabela.}\label{tab:tabela}
    \end{table}

    Write some text.
    Como mostrado na Tabela~\ref{tab:tabela}. \lipsum[5]

    Aqui está uma figura.
    Esta é a Figura~\ref{fig:figure}:
    \begin{figure}[htb!]
        \centering
        \includegraphics[width=0.5\textwidth]{figures/figure}
        \caption{Isto é um plot.}\label{fig:figure}
    \end{figure}

    \newpage

    Aqui está uma subfigura:

    \begin{figure}[htb!]
        \centering
        \begin{subfigure}[b]{0.45\textwidth}
             \centering
             \includegraphics[width=\textwidth]{figures/figure}
             \caption{$y=3\sin x$}
             \label{fig:three sin x}
         \end{subfigure}
         \hfill
         \begin{subfigure}[b]{0.45\textwidth}
             \centering
             \includegraphics[width=\textwidth]{figures/flores}
             \caption{$y=5/x$}
             \label{fig:five over x}
         \end{subfigure}
            \caption{Duas simples figuras.}
            \label{fig:twographs}
    \end{figure}

    A Figura~\ref{fig:twographs} é composta por duas subfiguras, a Figura~\ref{fig:three sin x} e a Figura~\ref{fig:five over x}.

    \section{Método}\label{sec:metodo}

    \lipsum[6-7]

    \section{Git e GitHub}\label{sec:outrometodo}

    Isto é um novo parágrafo, aqui iremos ver como controlar as versões dos documentos utilizando Git e iremos sincronizar este controlo de versões com o github.

    \subsection{Github}\label{subsec:github}

    Aqui submetemos tudo para o Github.

    \section{Equações}

    Com o \LaTeX é muito fácil escrever equações, numerá-las e referênciá-las com uma facilidade enorme.
    Aqui está um exemplo de uma equação não numerada:
    \[
        \phi(1) * \phi(2) = \Phi(f) = \int_{-\infty}^{+\infty}  \phi(t) + \phi(t+\tau) \cdot d\tau
    \]

    Aqui está o exemplo de uma equação numerada:
    \begin{equation}\label{eq:num-dist}
        N_g[\phi(t)] = \int_{-\infty}^{+\infty} g(t) \cdot \phi(t) \cdot dt
    \end{equation}

    Outro exemplo de uma equação com \emph{brackets}:
    \begin{equation}\label{eq:classe-c}
        \phi(t) \in C \, \text{ se } \,
        \begin{cases}
            \exists \phi^{(n)} (t) & \forall t,n \\
            \lim_{t \rightarrow +\infty} \left[t^n \cdot \phi(t)\right] = 0
        \end{cases}
    \end{equation}
%    \bibliography{main}
%    \bibliographystyle{plain}

\end{document}
